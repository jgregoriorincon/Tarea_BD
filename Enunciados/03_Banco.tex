La Policía quiere crear una base de datos sobre la seguridad en algunas entidades bancarias. Para ello tiene en cuenta:
\begin{itemize} 
\item Que cada entidad bancaria se caracteriza por un código y por el domicilio de su Central definido por una dirección.
\item Que cada entidad bancaria tiene más de una sucursal que también se caracteriza por un código y por el domicilio, así como por el número de empleados de dicha sucursal.
\item Que cada sucursal contrata, según el día, algunos vigilantes, que se caracterizan por un código y edad. Un vigilante puede ser contratado por diferentes sucursales (incluso, diferentes entidades), en distintas fechas y es un dato de interés dicha fecha, así como si se ha contratado con arma o no.
\item Por otra parte, se quiere controlar a las personas que han sido detenidas por atracar las sucursales de dichas entidades. Estas personas se definen por una clave (código) y su nombre completo.
\item Alguna de estas personas están integradas en algunas bandas organizadas y por ello se desea saber a qué banda pertenecen, sin ser de interés si la banda ha participado en el delito o no. Dichas bandas se definen por un número de banda y por el número de miembros.
\item Así mismo, es interesante saber en qué fecha ha atracado cada persona una sucursal.
Evidentemente, una persona puede atracar varias sucursales en diferentes fechas, así como que una sucursal puede ser atracada por varias personas.
\item Igualmente, se quiere saber qué Juez ha estado encargado del caso, sabiendo que un individuo, por diferentes delitos, puede ser juzgado por diferentes jueces.
\item Es de interés saber, en cada delito, si la persona detenida ha sido condenada o no y de haberlo sido, cuánto tiempo pasará en la cárcel. Un Juez se caracteriza por una clave interna del juzgado, su nombre y los años de servicio.
\end{itemize}
NOTA: En ningún caso interesa saber si un vigilante ha participado en la detención de un atracador. Es importante tener la capacidad de geo posicionar las sucursales bancarias con el fin de realizar análisis de densidad de robos por zonas dentro de períodos de tiempos. Las zonas serán definidas por las localidades de la ciudad