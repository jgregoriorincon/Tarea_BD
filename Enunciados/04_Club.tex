Un club náutico desea tener informatizados los datos correspondientes a sus instalaciones, empleados, socios y embarcaciones que se encuentran en dicho club. El club está organizado de la siguiente forma:
\begin{itemize}
    \item Los socios pertenecientes al club vienen definidos por su nombre, dirección, CC, teléfono y fecha de ingreso en el club.
    \item Las embarcaciones vienen definidas por: matricula, nombre, tipo y dimensiones.
    \item Los amarres tienen como datos de interés el número de amarre, la lectura del contador de agua y luz, y si tienen o no servicios de mantenimiento contratados.
    \item Por otro lado, hay que tener en cuenta que una embarcación pertenece a un socio aunque un socio puede tener varias embarcaciones. Una embarcación ocupará un amarre y un amarre está ocupado por una sola embarcación. Es importante la fecha en la que una embarcación es asignada a un amarre.
    \item Los socios pueden ser propietarios de amarres, siendo importante la fecha de compra del amarre. Hay que tener en cuenta que un amarre pertenece a un solo socio y que NO HAY ninguna relación directa entre la fecha en la que se compra un amarre y en la que una embarcación se asigna a un amarre.
    \item El club náutico está dividido en varias zonas definidas por una letra, el tipo de barcos que tiene, el número de barcos que contiene, la profundidad y el ancho de los amarres. Una zona tendrá varios amarres y un amarre pertenece a una sola zona.
    \item En cuanto a los empleados, estos vienen definidos por su código, nombre, dirección, teléfono y especialidad. Un empleado está asignado a varias zonas y en una zona puede haber más de un empleado, siendo de interés el número de barcos de los que se encarga en cada zona. Hay que tener en cuenta que un empleado puede no encargarse de todos los barcos de una zona.
\end{itemize}